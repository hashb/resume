%------------------------------------
% @brief    LaTeX2e Resume for Kamil K Wojcicki

%!TEX TS-program = xelatex
%!TEX encoding = UTF-8 Unicode

\documentclass[margin, line]{resume}
\usepackage[unicode=true,
            pdfpagelabels=false,
            pdfauthor={Kautilya Chenna},
            pdftitle={Kautilya Chenna's Resume created on \today},
            pdfsubject={Resume},
            pdfkeywords={Resume,
                         Robotics,
                         Machine Learning,
                         Motion Planning,
                         Trajectory Optimization,
                         ROS,
                         OpenCV,
                         TensorFlow,
                         PDF,
                         LaTeX}]{hyperref}

\hypersetup{
    colorlinks=true,
    linkcolor=cyan,
    filecolor=blue,
    urlcolor=[rgb]{0.01 0.01 0.5},
}

% FONTS
\usepackage{xltxtra}
\setmainfont[SmallCapsFont={Alegreya},
             SmallCapsFeatures={Letters=SmallCaps}]{Times}


\begin{document}
\author{Kautilya Chenna}
\name{\Huge \textsc{Kautilya Chenna}}
{
    \sc
    \hfill \href{mailto:chenna@outlook.com}{chenna@outlook.com}     \vspace{0mm}\\\vspace{0mm}%
    \hfill San Francisco Bay Area                                   \vspace{0mm}\\\vspace{-10.5mm}%
}

\begin{resume}

    \section{\mysidestyle \textsc{Summary}}
    I joined Omron Research as their 4th employee and have worked on a wide
    range of research projects mainly focused on Optimization for Motion
    Planning and Control. Before joining Omron, I spent 2 years at LL4MA Lab
    working on Deep Learning for robotic manipulation tasks.\\[1mm]
    I focus on delivering clean and maintainable code while iterating quickly.
    I am passionate about using technology to find solutions to our growing
    social needs.


    %%%%%%%%%%%%%%
    %%% Skills %%%
    %%%%%%%%%%%%%%

    \sectionline
    \section{\mysidestyle \textsc{Skills}}

    \textbf{Languages}: C++, Python, MATLAB \\[1mm]
    \textbf{Tools}: PCL, ROS, Gazebo, OpenCV, Tensorflow, Keras, Jenkins, Docker and k8s \\[1mm]
    \textbf{Robots}: Omron Adept Robots, KUKA LBR4+, Rethink Robotics Baxter, SimLab's Allegro Hand%


    %%%%%%%%%%%%%%%%%%
    %%% Experience %%%
    %%%%%%%%%%%%%%%%%%

    \sectionline
    \section{\mysidestyle \textsc{Experience}}

    \href{https://www.omron.com/global/americas/usa.html}{\textbf{Omron Research Center of America (ORCA)}} \vspace{1pt}\\\vspace{1pt}%
    \textsl{Robotics Software Engineer} \hfill \textbf{November 2018 -- Present}\\ \vspace{-4.5mm}
    \begin{list2}
        \item Developed test framework for an end-to-end random bin picking pipeline
        \item Designed and developed a real-time middleware for reactive robot control
        \item Created standards and guidelines for code reviews and code quality
        \item Built and maintained Jenkins CI infrastructure used by 4 teams
        \item Saved the organization >\$150k by reviewing and selecting an 
        opensource \\alternative to a commercial product
    \end{list2}\vspace{-2.25mm}

    \href{https://robot-learning.cs.utah.edu}{\textbf{Learning Lab for Manipulation Autonomy (LL4MA Lab)}}, University of Utah \vspace{1pt}\\\vspace{1pt}%
    \textsl{Graduate Research Assistant} \hfill \textbf{August 2016 -- 2018}\\ \vspace{-4.5mm}
    \begin{list2}
        \item Built a fast object detection and tracking pipeline, which is
        used by multiple teams in the Lab.
        \item Implemented Grasp Controllers and end-to-end Grasping Pipelines
        with motion planning and execution.
        \item Created and maintained a ROS wrapper for Blensor, a kinect camera
        simulator with realistic error models
    \end{list2}\vspace{-0.1mm}

    %    \href{http://www.aero.iisc.ernet.in/~dinesh/web/}{\textbf{NMCAD Lab}}, Indian Institute of Science \vspace{1pt}\\\vspace{1pt}%
    %    \textsl{Research Intern} \hfill \textbf{January 2015 -- July 2016}\\ \vspace{-4.5mm}
    %    \begin{list2}
    %    	\item Worked on the design and fabrication of a Flapping Wing Micro Aerial Vehicle (MAV).
    %    	\item Developed autonomous navigation and collision checking algorithms for the MAV.
    %    \end{list2}\vspace{-0.1mm}


    %%%%%%%%%%%%%%%%%
    %%% Education %%%
    %%%%%%%%%%%%%%%%%

    \sectionline
    \section{\mysidestyle \textsc{Education}}
    \textbf{University of Utah}, Salt Lake City, Utah %\hfill \texttt{GPA: 3.40}
    \\\vspace{1mm}%
    \textsl{Master of Science in Mechanical Engineering} \hfill \textbf{ Aug 2016 -- Aug 2018}\\%
    % \begin{list2}
    % 	\item Expected to graduate: Spring 2018
    % 	\item Advisor:  \href{https://www.cs.utah.edu/~thermans/}{Dr. Tucker Hermans}
    % \end{list2}\vspace{-1.5mm}
    \textbf{BMS College of Engineering}, Bangalore, India %\hfill \texttt{GPA: 3.52}
    \\\vspace{1mm}%
    \textsl{Bachelor of Engineering in Mechanical Engineering } \hfill \textbf{Sept 2011 -- May 2015}\\%
    % \begin{list2}
    % 	\item Graduated with First Class Distinction.
    % \end{list2}\vspace{-1.5mm}
    \textbf{Relevant Coursework:} 3D Computer Vision, Artificial Intelligence,
    Convex Optimization, Intro to Robotics, Intro to Robot Control,
    Motion Planning, Machine Learning, Probabilistic Modeling, and
    System Identification.


    %%%%%%%%%%%%%%%%%%%%
    %%% Publications %%%
    %%%%%%%%%%%%%%%%%%%%

    \sectionline
    \section{\mysidestyle \textsc{Publications}}
     ``Planning Multi-Fingered Grasps as Probabilistic Inference in a Learned Deep Network''; Qingkai Lu,\\
     \mbox{\bf Kautilya Chenna}, Balakumar Sundaralingam, Tucker Hermans; \textit{International Symposium on\\ Robotics Research (ISRR),} 2017. \texttt{[\href{http://www.cs.utah.edu/~thermans/papers/lu-isrr2017-deep-multifinger-grasping.pdf}{PDF}]} \texttt{[\href{https://robot-learning.cs.utah.edu/project/grasp\_inference}{CODE}]}%


    %%%%%%%%%%%%%%%%
    %%% Projects %%%
    %%%%%%%%%%%%%%%%

    \sectionline
    \section{\mysidestyle \textsc{Selected \\ Projects}}

    \textbf{Object Detection and Segmentation in Point Cloud data using PointNet} \hfill \textbf{January 2018}
    \begin{list2}
        \item Trained modified \textbf{PointNet} model on \textbf{YCB object dataset} and \textbf{BigBird dataset}.
        \item Model runs at \textbf{24 fps} on a NVIDIA GeForce 1060 GPU with an accuracy of 88.3\%.
    \end{list2}\vspace{-3.2mm}

    \textbf{Grasp Collision detection using Convolutional Neural Networks} \hfill \textbf{Ongoing}
    \begin{list2}
        \item Developed a CNN model to detect collisions btw robot and environment using PointClouds and JointState.
        \item Model classifies collisions with an \textbf{accuracy of 84.7\%} and is \textasciitilde 30\% faster than FCL.
    \end{list2}\vspace{-3.2mm}

    \textbf{Video Action recognition using Deep Learning}\hfill \textbf{October 2017}
    \begin{list2}
    \item Implemented a \textbf{Bi-Directional LSTM Model} on \textbf{VGG16} Net using Keras to classify actions in scenes.
    \item Achieved a Mean Average Precision of \textbf{15.7 mAP} compared to the State of the Art of 21.4 mAP.
    \end{list2}\vspace{-3.2mm}

    \textbf{Autonomous Grasp Inference and Execution using Baxter and KUKA lwr4 Robots}\hfill \textbf{January 2017}
    \begin{list2}
        \item Designed an end-to-end grasping pipeline to grasp objects on a table autonomously.
        \item Training data was collected in Gazebo simulation and tested in real world. [\href{https://robot-learning.cs.utah.edu/project/grasp\_inference}{ISRR 2017}]
    \end{list2}\vspace{-3.2mm}

    \textbf{Others:} Motion Planning: TrajOpt, RRT and Variants, RealTime RRT*; Image Segmentation with GMM, Image De-noising using MRF;


    %%%%%%%%%%%%%%%%%%%%%%%%%%%
    %%% Contact Information %%%
    %%%%%%%%%%%%%%%%%%%%%%%%%%%
%
%%	\sectionseperator
%	\section{\mysidestyle \textsc{Contact \\ Information}}
%
%	\texttt{San Francisco Bay Area}                                  \hfill phone: \texttt{\href{tel:13855551234}{+1 (385) 555-xxxx }}          \vspace{0mm}\\\vspace{0mm}%
%	\texttt{California {\textemdash} 945xx}                          \hfill email: \texttt{\href{mailto:chenna@outlook.com}{chenna@outlook.com}}       %


    %%%%%%%%%%%%%
    %%% Links %%%
    %%%%%%%%%%%%%

    \sectionline
    \section{\mysidestyle \textsc{Links}}
    \textbf{Website}: \texttt{\href{https://chenna.me}{https://chenna.me}} \hspace{2mm}%\\[1mm]
    \textbf{Linkedin}: \texttt{\href{https://www.linkedin.com/in/kautilyachenna/}{kautilyachenna}} \hspace{2mm}%\\[1mm]
    \textbf{Github}: \texttt{\href{https://github.com/hashb}{hashb}}
    \vspace{-3.2mm}

\begin{center}
    {\scriptsize  Last updated: \today\- •\- Typeset in {\fontspec{Alegreya}\XeTeX }\\
        \href{https://chenna.me/resume}{https://chenna.me/resume}}
\end{center}

\end{resume}
\end{document}

% EOF

