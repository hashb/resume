%------------------------------------
% @brief    LaTeX2e Resume for Kamil K Wojcicki

%!TEX TS-program = xelatex
%!TEX encoding = UTF-8

\documentclass[letterpaper, margin, line, 10.5pt]{resume}
\usepackage{hyperref}
\usepackage{xcolor}
\hypersetup{
	colorlinks=true,
	linkcolor=cyan,
	filecolor=blue,      
	urlcolor=magenta,
}

% FONTS
\usepackage{xunicode}
\usepackage{xltxtra}
\defaultfontfeatures{Mapping=tex-text}
\setromanfont [Ligatures={Common}, Numbers={OldStyle}, Variant=01]{Times New Roman}
\setmonofont[Scale=0.8]{Monaco}


\begin{document}
\author{Kautilya Chenna}
\name{\Large Kautilya Chenna}
% \address{1495 E 100 S rm 1550 MEK\\
% Salt Lake City, Utah 84112}
% \email{chenna@outlook.com}
% \phone{(385) 528-7547}
% \webpage{https://chenna.me}

\begin{resume}

	%%%%%%%%%%%%%%%%%%%%%%%%%%%
    %%% Contact Information %%%
    %%%%%%%%%%%%%%%%%%%%%%%%%%%
    
    \section{\myheadingstyle Contact \\ Information}

    \texttt{7605 S Quatar Way}                            \hfill phone: \texttt{\href{tel:13855287547}{+1 (385) 528-7547}}          \vspace{0mm}\\\vspace{0mm}%
    \texttt{Aurora, Colorado 80016}                          \hfill email: \texttt{\href{mailto:chenna@outlook.com}{chenna@outlook.com}}       %


    %%%%%%%%%%%%%%
    %%% Skills %%%
    %%%%%%%%%%%%%%
    
	\sectionseperator
	\section{\myheadingstyle Skills} 
	
	\textbf{Languages}: C++, Python, MATLAB. \\[1mm]
	\textbf{Tools}: PCL, ROS, Gazebo, OpenCV, Tensorflow, Blender, Keras. \\[1mm]
	% \textbf{Design Tools}: SolidWorks, PTC Creo Parametric, Autodesk Inventor, ANSYS. \\[1mm]
	\textbf{Robots}: KUKA LBR4+, Rethink Robotics Baxter, SimLab's Allegro Hand, Quanser HD2.%


	%%%%%%%%%%%%%%%%%
    %%% Education %%%
    %%%%%%%%%%%%%%%%%
    
    \sectionseperator
    \section{\myheadingstyle Education}
    \textbf{University of Utah}, Salt Lake City, Utah \hfill \texttt{GPA: 3.40}
    \\\vspace{1mm}%
    \textsl{Master of Science in Robotics} \hfill \textbf{ Aug 2016 -- Aug 2018}\\%
	% \begin{list2}
	% 	\item Expected to graduate: Spring 2018
	% 	\item Advisor:  \href{https://www.cs.utah.edu/~thermans/}{Dr. Tucker Hermans}
	% \end{list2}\vspace{-1.5mm}
    \textbf{BMS College of Engineering}, Bangalore, India \hfill \texttt{GPA: 3.52}
    \\\vspace{1mm}%
    \textsl{Bachelor of Engineering in Mechanical Engineering (Robotics)} \hfill \textbf{Sept 2011 -- May 2015}\\%
	% \begin{list2}
	% 	\item Graduated with First Class Distinction.
	% \end{list2}\vspace{-1.5mm}
	\textbf{Relevant Coursework:} Probabilistic Modeling, 3D Computer Vision, Artificial Intelligence, Motion Planning, Machine Learning, Robotics, Robot Control and System Identification.

	%%%%%%%%%%%%%%%%%%
    %%% Coursework %%%
    %%%%%%%%%%%%%%%%%%
    
    % \sectionseperator
	% \section{\myheadingstyle Coursework}
	% Probabilistic Modeling, 3D Computer Vision, Artificial Intelligence, Motion Planning, Machine Learning, Robotics and System Identification.%


    %%%%%%%%%%%%%%%%%%%%
    %%% Publications %%%
    %%%%%%%%%%%%%%%%%%%%
    
    \sectionseperator
    \section{\myheadingstyle Publications}
     ``Planning Multi-Fingered Grasps as Probabilistic Inference in a Learned Deep Network''; Qingkai Lu, \\
     \mbox{\bf Kautilya Chenna}, Balakumar Sundaralingam, Tucker Hermans; \textit{International Symposium on Robotics Research (ISRR),} 2017. \texttt{[\href{http://www.cs.utah.edu/~thermans/papers/lu-isrr2017-deep-multifinger-grasping.pdf}{PDF}]} \texttt{[\href{https://robot-learning.cs.utah.edu/project/grasp\_inference}{CODE}]}%


    %%%%%%%%%%%%%%%%%%
    %%% Experience %%%
    %%%%%%%%%%%%%%%%%%
    
    \sectionseperator
    \section{\myheadingstyle Experience}

    \href{https://robot-learning.cs.utah.edu}{\textbf{Learning Lab for Manipulation Autonomy (LL4MA Lab)}}, University of Utah \vspace{1pt}\\\vspace{1pt}%
    \textsl{Graduate Research Assistant} \hfill \textbf{August 2016 -- present}\\ \vspace{-4.5mm}
	\begin{list2}
		\item Built a fast object detection and tracking pipeline, which is used by multiple teams in the Lab.
		\item Implemented Grasp Controllers and end-to-end Grasping Pipelines with motion planning and execution.
	\end{list2}\vspace{-2.25mm}

    \href{http://www.aero.iisc.ernet.in/~dinesh/web/}{\textbf{NMCAD Lab}}, Indian Institute of Science \vspace{1pt}\\\vspace{1pt}%
    \textsl{Research Intern} \hfill \textbf{January 2015 -- July 2016}\\ \vspace{-4.5mm}
    \begin{list2}
    	\item Worked on the design and fabrication of a Flapping Wing Micro Aerial Vehicle (MAV).
    	\item Developed autonomous navigation and collision checking algorithms for the MAV.
    \end{list2}\vspace{-0.5mm}


    %%%%%%%%%%%%%%%%
	%%% Projects %%%
    %%%%%%%%%%%%%%%%
    
	\sectionseperator
	\section{\myheadingstyle Selected \\ Projects}
	\textbf{Extrinsic Calibration of Stereo Camera and Velodyne LiDAR} \hfill \textbf{June 2018}
	\begin{list2}
		\item Developed a ROS package to automate calibration between Velodyne VLP-16 and ZED stereo camera.
		\item Reduced the mean point to point error by \textbf{72\%} compared to manual feature based calibration.
	\end{list2}\vspace{-3.2mm}
	
	\textbf{Real-time Semantic Segmentation on Low-Power Android Devices} \hfill \textbf{May 2018}
	\begin{list2}
		\item Developed a fast background subtraction for portrait video based on modified \textbf{SegNet} model.
		\item Model achieved a \textbf{mean IoU of 87.3\% at 30 FPS} on Google Pixel 2.
	\end{list2}\vspace{-3.2mm}

	\textbf{Estimating Depth from a single image using FCN Network} \hfill \textbf{March 2018}
	\begin{list2}
		\item Implemented a modified \textbf{FCN Net} and trained it on NYU Depth Dataset and KITTI Dataset.
		\item Model achieved a mean \textbf{RMSE error of 0.294} on NYU Depth and \textbf{0.312} on KITTI Dataset.
	\end{list2}\vspace{-3.2mm}

	\textbf{Object Detection and Segmentation in Point Cloud data using PointNet} \hfill \textbf{January 2018}
	\begin{list2}
		\item Trained modified \textbf{PointNet} model on \textbf{YCB object dataset} and \textbf{BigBird dataset}.
		\item Model runs at \textbf{24 fps} on a NVIDIA GeForce 1060 GPU with an accuracy of 88.3\%.
	\end{list2}\vspace{-3.2mm}

	\textbf{Grasp Collision detection using Convolutional Neural Networks} \hfill \textbf{Ongoing}
	\begin{list2}
		\item Developed a CNN model to detect collisions btw robot and environment using PointClouds and JointState.
		\item Model classifies collisions with an \textbf{accuracy of 84.7\%} and is \textasciitilde 30\% faster than FCL.
	\end{list2}\vspace{-3.2mm}
	
	\textbf{The Search for Twitter Spam Bots} \hfill \textbf{December 2017}
	\begin{list2}
		\item Implemented a machine learning algorithms from scratch to predict if a twitter user's content is spam.
		\item \textbf{Boosted trees} achieved an accuracy of \textbf{97\%} and \textbf{ranked 1st} in \href{https://www.kaggle.com/c/uofu-ml-fall-2017/leaderboard}{Kaggle competition}.
	\end{list2}\vspace{-3.2mm}

	\textbf{Video Action recognition using Deep Learning}\hfill \textbf{October 2017}
	\begin{list2}
	\item Implemented a \textbf{Bi-Directional LSTM Model} on \textbf{VGG16} Net using Keras to classify actions in scenes.
	\item Achieved a Mean Average Precision of \textbf{15.7 mAP} compared to the State of the Art of 21.4 mAP.
	\end{list2}\vspace{-3.2mm}

	\textbf{Autonomous Grasp Inference and Execution using Baxter and KUKA lwr4 Robots}\hfill \textbf{January 2017}
	\begin{list2}
		\item Designed an end-to-end grasping pipeline to grasp objects on a table autonomously.
		\item Training data was collected in Gazebo simulation and tested in real world. [\href{https://robot-learning.cs.utah.edu/project/grasp\_inference}{ISRR 2017}]
	\end{list2}\vspace{-3.2mm}

	\textbf{Others:} Motion Planning: TrajOpt, RRT and Variants, RealTime RRT*; Image Segmentation with GMM, Image De-noising using MRF; 
	

	%%%%%%%%%%%%%
	%%% Links %%%
	%%%%%%%%%%%%%
	
	\sectionseperator
    \section{\myheadingstyle Links}
	\textbf{Website}: \texttt{\href{https://chenna.me}{https://chenna.me}} \hspace{2mm}%\\[1mm]
    \textbf{Linkedin}: \texttt{\href{https://www.linkedin.com/in/kautilyachenna/}{kautilyachenna}} \hspace{2mm}%\\[1mm]
    \textbf{Github}: \texttt{\href{https://github.com/hashb}{hashb}}
    \vspace{-3.2mm}

\begin{center}
	{\scriptsize  Last updated: \today\- •\- Typeset in \href{}{
			\fontspec{Alegreya}\XeTeX }\\
		\href{http://chenna.me/resume}{http://chenna.me/resume}}
\end{center}

\end{resume}
\end{document}

% EOF

