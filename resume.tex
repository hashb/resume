%------------------------------------
% @brief    LaTeX2e Resume for Kamil K Wojcicki

%!TEX TS-program = xelatex
%!TEX encoding = UTF-8

\documentclass[margin, line]{resume}
\usepackage[pdfpagelabels=false,
			pdfauthor={Kautilya Chenna},
			pdftitle={Kautilya Chenna's Resume created on \today},
			pdfsubject={Resume},
			pdfkeywords={Resume,
						 Robotics,
						 Machine Learning,
						 Motion Planning,
						 Trajectory Optimization,
						 ROS,
						 OpenCV,
						 TensorFlow,
						 PDF,
						 LaTeX}]{hyperref}

\usepackage{xcolor}
\hypersetup{
	colorlinks=true,
	linkcolor=cyan,
	filecolor=blue,      
	urlcolor=[rgb]{0.8 0 0},
}

% FONTS
%\usepackage{xunicode}
\usepackage{xltxtra}
\setmainfont[SmallCapsFont={Alegreya},
			 SmallCapsFeatures={Letters=SmallCaps}]{Georgia}


\begin{document}
\author{Kautilya Chenna}
\name{\Huge \textsc{Kautilya Chenna}}
{
	\sc
	\hfill \href{mailto:chenna@outlook.com}{chenna@outlook.com}		\vspace{0mm}\\\vspace{0mm}%
	\hfill San Francisco, CA										\vspace{0mm}\\\vspace{-10.5mm}%
}

\begin{resume}

	\section{\mysidestyle \textsc{Summary}}
	lorem ipsum 


    %%%%%%%%%%%%%%
    %%% Skills %%%
    %%%%%%%%%%%%%%
    
    \sectionline
	\section{\mysidestyle \textsc{Skills}} 
	
	\textbf{Languages}: C++, Python, MATLAB. \\[1mm]
	\textbf{Tools}: PCL, ROS, Gazebo, OpenCV, Tensorflow, Keras. \\[1mm]
	\textbf{Robots}: KUKA LBR4+, Rethink Robotics Baxter, SimLab's Allegro Hand, Quanser HD2.%


    %%%%%%%%%%%%%%%%%%
	%%% Experience %%%
	%%%%%%%%%%%%%%%%%%
	
	\sectionline
	\section{\mysidestyle \textsc{Experience}}
	
	\href{https://www.omron.com/global/americas/usa.html}{\textbf{Omron Research Center of America (ORCA)}} \vspace{1pt}\\\vspace{1pt}%
	\textsl{Motion Planning Engineer} \hfill \textbf{November 2018 -- Present}\\ \vspace{-4.5mm}
	\begin{list2}
		\item Motion Planning and Grasp Planning for Random Bin Picking
		\item Trajectory Optimization and more stuff. Ssshhhhh....
	\end{list2}\vspace{-2.25mm}
	
	\href{https://robot-learning.cs.utah.edu}{\textbf{Learning Lab for Manipulation Autonomy (LL4MA Lab)}}, University of Utah \vspace{1pt}\\\vspace{1pt}%
	\textsl{Graduate Research Assistant} \hfill \textbf{August 2016 -- 2018}\\ \vspace{-4.5mm}
	\begin{list2}
		\item Built a fast object detection and tracking pipeline, which is used by multiple teams in the Lab.
		\item Implemented Grasp Controllers and end-to-end Grasping Pipelines with motion planning and execution.
	\end{list2}\vspace{-0.1mm}
	
	%    \href{http://www.aero.iisc.ernet.in/~dinesh/web/}{\textbf{NMCAD Lab}}, Indian Institute of Science \vspace{1pt}\\\vspace{1pt}%
	%    \textsl{Research Intern} \hfill \textbf{January 2015 -- July 2016}\\ \vspace{-4.5mm}
	%    \begin{list2}
	%    	\item Worked on the design and fabrication of a Flapping Wing Micro Aerial Vehicle (MAV).
	%    	\item Developed autonomous navigation and collision checking algorithms for the MAV.
	%    \end{list2}\vspace{-0.1mm}


	%%%%%%%%%%%%%%%%%
    %%% Education %%%
    %%%%%%%%%%%%%%%%%
    
	\sectionline
    \section{\mysidestyle \textsc{Education}}
    \textbf{University of Utah}, Salt Lake City, Utah %\hfill \texttt{GPA: 3.40}
    \\\vspace{1mm}%
    \textsl{Master of Science in Robotics} \hfill \textbf{ Aug 2016 -- Aug 2018}\\%
	% \begin{list2}
	% 	\item Expected to graduate: Spring 2018
	% 	\item Advisor:  \href{https://www.cs.utah.edu/~thermans/}{Dr. Tucker Hermans}
	% \end{list2}\vspace{-1.5mm}
    \textbf{BMS College of Engineering}, Bangalore, India %\hfill \texttt{GPA: 3.52}
    \\\vspace{1mm}%
    \textsl{Bachelor of Engineering in Mechanical Engineering (Robotics)} \hfill \textbf{Sept 2011 -- May 2015}\\%
	% \begin{list2}
	% 	\item Graduated with First Class Distinction.
	% \end{list2}\vspace{-1.5mm}
	\textbf{Relevant Coursework:} Probabilistic Modeling, 3D Computer Vision, Artificial Intelligence, Motion Planning, Machine Learning, Convex Optimization, Robotics, Robot Control and System Identification.


    %%%%%%%%%%%%%%%%%%%%
    %%% Publications %%%
    %%%%%%%%%%%%%%%%%%%%
    
	\sectionline
    \section{\mysidestyle \textsc{Publications}}
     ``Planning Multi-Fingered Grasps as Probabilistic Inference in a Learned Deep Network''; Qingkai Lu,
     \mbox{\bf Kautilya Chenna}, Balakumar Sundaralingam, Tucker Hermans; \textit{International Symposium on\\ Robotics Research (ISRR),} 2017. \texttt{[\href{http://www.cs.utah.edu/~thermans/papers/lu-isrr2017-deep-multifinger-grasping.pdf}{PDF}]} \texttt{[\href{https://robot-learning.cs.utah.edu/project/grasp\_inference}{CODE}]}%


    %%%%%%%%%%%%%%%%
	%%% Projects %%%
    %%%%%%%%%%%%%%%%
    
	\sectionline
	\section{\mysidestyle \textsc{Selected \\ Projects}}

	\textbf{Others:} Motion Planning: TrajOpt, RRT and Variants, RealTime RRT*; Image Segmentation with GMM, Image De-noising using MRF; 
	
	
	%%%%%%%%%%%%%%%%%%%%%%%%%%%
	%%% Contact Information %%%
	%%%%%%%%%%%%%%%%%%%%%%%%%%%
%	
%%	\sectionseperator
%	\section{\mysidestyle \textsc{Contact \\ Information}}
%	
%	\texttt{San Francisco Bay Area}                            \hfill phone: \texttt{\href{tel:13855551234}{+1 (385) 555-xxxx }}          \vspace{0mm}\\\vspace{0mm}%
%	\texttt{California {\textemdash} 945xx}                          \hfill email: \texttt{\href{mailto:chenna@outlook.com}{chenna@outlook.com}}       %


	%%%%%%%%%%%%%
	%%% Links %%%
	%%%%%%%%%%%%%
	
	\sectionline
    \section{\mysidestyle \textsc{Links}}
	\textbf{Website}: \texttt{\href{https://chenna.me}{https://chenna.me}} \hspace{2mm}%\\[1mm]
    \textbf{Linkedin}: \texttt{\href{https://www.linkedin.com/in/kautilyachenna/}{kautilyachenna}} \hspace{2mm}%\\[1mm]
    \textbf{Github}: \texttt{\href{https://github.com/hashb}{hashb}}
    \vspace{-3.2mm}

\begin{center}
	{\scriptsize  Last updated: \today\- •\- Typeset in {\fontspec{Alegreya}\XeTeX }\\
		\href{https://chenna.me/resume}{https://chenna.me/resume}}
\end{center}

\end{resume}
\end{document}

% EOF

