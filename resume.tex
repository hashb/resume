%------------------------------------
% @brief    LaTeX2e Resume for Kamil K Wojcicki

%!TEX TS-program = xelatex
%!TEX encoding = UTF-8 Unicode

\documentclass[letterpaper, margin, line]{resume}
\usepackage{hyperref}
\usepackage{xcolor}
\hypersetup{
	colorlinks,
	linkcolor={red!30!black},
	citecolor={blue!50!black},
	urlcolor={blue!70!black}
}

% FONTS
\usepackage{xunicode}
\usepackage{xltxtra}
\defaultfontfeatures{Mapping=tex-text}
\setromanfont [Ligatures={Common}, Numbers={OldStyle}, Variant=01]{Times New Roman}
\setmonofont[Scale=0.8]{Monaco}


\begin{document}
\author{Kautilya Chenna}
\name{\Large Kautilya Chenna}
%\email{chenna@outlook.com}
%\phone{(385) 528-7547}
%\webpage{https://chenna.me}

\begin{resume}

    % Contact Information
    \section{\mysidestyle Contact\\Information}

    525 S 900 E, Apt C2                            \hfill phone: \texttt{\href{tel:13855287547}{+1 (385) 528-7547}}          \vspace{0mm}\\\vspace{0mm}%
    Salt Lake City, Utah \texttt{84102}                          \hfill email: \texttt{\href{mailto:chenna@outlook.com}{chenna@outlook.com}}          \vspace{1pt}%\\\vspace{1pt}%

    % Research Interests
    \section{\mysidestyle Research\\Interests}

    Robotics: Perception, Manipulation and Cognition; Machine Learning, Computer Vision.
    \vspace{1pt}

    % Education
    \section{\mysidestyle Education}

    \textbf{University of Utah}, Salt Lake City, Utah
    \vspace{2mm}\\\vspace{1mm}%
    \textsl{Master of Science in Engineering, Robotics} \hfill \textbf{ August 2016 -- present}\vspace{-3mm}\\\vspace{-1mm}%
    \begin{list2}
        \item Expected graduation date: May 2018
        \item Advisors:  \href{https://www.cs.utah.edu/~thermans/}{Dr. Tucker Hermans}
    \end{list2}\vspace{-1.5mm}
    \textbf{BMS College of Engineering}, Bangalore, India
    \vspace{2mm}\\\vspace{1mm}%
    \textsl{Bachelor of Engineering in Mechanical Engineering} \hfill \textbf{September 2011 -- May 2015}\vspace{-3mm}\\\vspace{-1mm}%
%	\begin{list2}
%		\item Graduated with First Class Distinction.
%	\end{list2}\vspace{-1.5mm}

    % Honours and Awards
	% ** crickets **

    % Publications
    \section{\mysidestyle Publications}
     ``Planning Multi-Fingered Grasps as Probabilistic Inference in a Learned Deep Network''; Qingkai Lu, \\
     \mbox{\bf Kautilya Chenna}, Balakumar Sundaralingam, Tucker Hermans; \textit{International Symposium on Robotics Research (ISRR),} 2017. \texttt{[\href{http://www.cs.utah.edu/~thermans/papers/lu-isrr2017-deep-multifinger-grasping.pdf}{PDF}]}

    % Professional Experience
    \section{\mysidestyle Experience}

    \href{https://robot-learning.cs.utah.edu}{\textbf{Learning Lab for Manipulation Autonomy (LL4MA)}}, University of Utah \vspace{1pt}\\\vspace{1pt}%
    \textsl{Graduate Research Assistant} \hfill \textbf{August 2016 -- present}\\
Currently working under Dr. Tucker Hermans on developing a machine learning algorithm that predicts if the robot will be in collision for a given configuration using only pointcloud data and joint states.


    \href{http://www.aero.iisc.ernet.in/~dinesh/web/}{\textbf{NMCAD Lab}}, Indian Institute of Science \vspace{1pt}\\\vspace{1pt}%
    \textsl{Project Assistant} \hfill \textbf{January 2015 -- July 2016}\\
    Worked under Prof. Dineshkumar Harursampath on the project \textit{``Design and Fabrication of a Conventional Flapping Wing Micro Aerial Vehicle.''} We worked towards developing a platform for testing various wing designs, materials and mechanisms on the MAV.

    
    %__________________________________________________________________________________________________________________
	% Projects
	\section{\mysidestyle Selected\\Projects}
	
	\textbf{Baxter Grasp Pipeline}\hfill \textbf{January 2017}
    \begin{list2}
	\item Developed a grasping pipeline to grasp objects on a table.
	\item Tools Used: PCL, ROS, Moveit, Graspit, tensorflow
	\end{list2}\vspace{-1.5mm}

	\textbf{Video Action recognition using Deep Learning}\hfill \textbf{October 2016}
	\begin{list2}
	\item Implemented a Deep Neural Network using tensorflow to classify actions in scenes.
	\item Achieved a mean average precision of 15.4\% on the Charades Dataset.
	\end{list2}
	


    %__________________________________________________________________________________________________________________
    % Skills
    \section{\mysidestyle Skills} 

    \textbf{Languages}: Python, MATLAB, C++, Java. \\[1mm]
    \textbf{Tools}: PCL (Pointcloud Library), ROS (Robot Operating System), OpenCV, Tensorflow, Blender (3D Graphics), Keras, Graspit Simulator, Gazebo, V-REP. \\[1mm]
    \textbf{Design Tools}: SolidWorks, PTC Creo Parametric, Autodesk Inventor, ANSYS. \\[1mm]
    \textbf{Robots}: KUKA LBR4, Rethink Robotics Baxter, SimLab's Allegro Hand , Quanser HD2
    
    % Relevant Coursework
    \section{\mysidestyle Relevant\\Coursework}
    Robot Kinematics and Dynamics, Controls (Linear, Nonlinear, and Embedded), Computer Vision, Artificial Intelligence, Motion Planning, Machine Learning, Probabilistic Modeling, System ID for Robotics.
    
    
    \section{\mysidestyle Links}
	\textbf{Website}: \texttt{\href{https://chenna.me}{https://chenna.me}} \\[1mm]
    \textbf{Linkedin}: \texttt{\href{https://www.linkedin.com/in/kautilyachenna/}{https://www.linkedin.com/in/kautilyachenna/}}\\[1mm]
    \textbf{Github}: \texttt{\href{https://github.com/hashb}{https://github.com/hashb}}
    

\begin{center}
	{\scriptsize  Last updated: \today\- •\- Typeset in \href{}{
			\fontspec{Times New Roman}\XeTeX }\\
		\href{http://chenna.me/resume}{http://chenna.me/resume}}
\end{center}

\end{resume}
\end{document}

% EOF

